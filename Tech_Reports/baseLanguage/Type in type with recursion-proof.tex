%% LyX 2.3.0 created this file.  For more info, see http://www.lyx.org/.
%% Do not edit unless you really know what you are doing.
\documentclass[english]{article}
\usepackage[T1]{fontenc}
\usepackage[latin9]{inputenc}
\usepackage{mathtools}
\usepackage{amsmath}
\usepackage{amssymb}
\usepackage{babel}
\begin{document}

\title{Type Soundness in an Intensional Dependent Type Theory with Type-in-Type
and Recursion}
\maketitle

\section{Examples}

logical unsoundness:

$\mathsf{fun}\,f:\left(x.x\right).\,x:\star.f\,x\qquad:\varPi x:\star.x$

\subsection{some constructs}

While logically unsound, the language is extremely expressive. The
following calculus of Constructions constructs are expressible,

$a_{1}=_{A}a_{2}\coloneqq\lambda A:\star.\lambda a_{1}:A.\lambda a_{2}:A.\varPi C:\left(A\rightarrow\star\right).C\,a_{1}\rightarrow C\,a_{2}$

$refl_{a:A}\coloneqq\lambda C:\left(A\rightarrow\star\right).\lambda x:C\,a.x\qquad:a=_{A}a$

$Unit\coloneqq\varPi A:\star.A\rightarrow A$

$tt\coloneqq\lambda A:\star.\lambda a:A.a$

$\perp\coloneqq\varPi x:\star.x$

$\lnot A\coloneqq\varPi A:\star..A\rightarrow\perp$

\subsubsection{Church Booleans}

$\mathbb{B}_{c}\coloneqq\varPi A:\star.A\rightarrow A\rightarrow A$

$true_{c}\coloneqq\lambda A:\star.\lambda then:A.\lambda else:A.then$

$false_{c}\coloneqq\lambda A:\star.\lambda then:A.\lambda else:A.else$

\subsubsection{Church $\mathbb{N}$}

$\mathbb{N}_{c}\coloneqq\varPi A:\star.(A\rightarrow A)\rightarrow A\rightarrow A$

$0_{c}\coloneqq\lambda A:\star.\lambda s:(A\rightarrow A).\lambda z:A.z$

$1_{c}\coloneqq\lambda A:\star.\lambda s:(A\rightarrow A).\lambda z:A.s\,z$

$2_{c}\coloneqq\lambda A:\star.\lambda s:(A\rightarrow A).\lambda z:A.s\,\left(s\,z\right)$

...

\subsubsection{Large Elimination}

Since there is type-in-type, a kind of large elimination is possible

$\lambda b:\mathbb{B}_{c}.b\,\star\,Unit\,\perp$

$\lambda n:\mathbb{N}_{c}.n\,\star\,(\lambda-.Unit)\,\perp$

thus $\lnot1_{c}=_{\mathbb{N}_{c}}0_{c}$ is provable (in a non trivial
way):

$\lambda pr:\left(\varPi C:\left(\mathbb{N}_{c}\rightarrow\star\right).C\,1_{c}\rightarrow C\,0_{c}\right).pr\,\left(\lambda n:\mathbb{N}_{c}.n\,\star\,(\lambda-.Unit)\,\perp\right)\,tt\qquad:\lnot1_{c}=_{\mathbb{N}_{c}}0_{c}$

$\lnot true_{c}=_{\mathbb{B}_{c}}false_{c}$ is provable:

$\lambda pr:\left(\varPi C:\left(\mathbb{B}_{c}\rightarrow\star\right).C\,true_{c}\rightarrow C\,false_{c}\right).pr\,\left(\lambda b:\mathbb{B}_{c}.b\,\star\,Unit\,\perp\right)\,tt\qquad:\lnot true_{c}=_{\mathbb{B}_{c}}false_{c}$

$\lnot Unit=_{\star}\perp$ is provable:

$\lambda pr:\left(\varPi C:\left(\star\rightarrow\star\right).C\,Unit\rightarrow C\,\perp\right).pr\,\left(\lambda x.x\right)tt\qquad:\lnot Unit=_{\star}\perp$

$\lnot\star=_{\star}\perp$ is provable:

$\lambda pr:\left(\varPi C:\left(\star\rightarrow\star\right).C\,\star\rightarrow C\,\perp\right).pr\,\left(\lambda x.x\right)\,\perp\qquad:\lnot\star=_{\star}\perp$

There are more examples in \cite{cardelli1986polymorphic} where Cardelli
has studied a similar system.

\section{Properties}

\subsection{Contexts}

\subsubsection{Sub-Contexts are well formed}

The following rules are admissible:

\[
\frac{\Gamma,\Gamma'\vdash}{\Gamma\vdash}
\]

\[
\frac{\Gamma,\Gamma'\vdash M:\sigma}{\Gamma\vdash}
\]

\[
\frac{\Gamma,\Gamma'\vdash M\Rrightarrow M':\sigma}{\Gamma\vdash}
\]

\[
\frac{\Gamma,\Gamma'\vdash M\Rrightarrow_{\ast}M':\sigma}{\Gamma\vdash}
\]

\[
\frac{\Gamma,\Gamma'\vdash M\equiv M':\sigma}{\Gamma\vdash}
\]

by mutual induction on the derivations.

\subsubsection{Context weakening}

For any derivation of $\Gamma\vdash\sigma:\star$, the following rules
are admissible:

\[
\frac{\Gamma,\Gamma'\vdash}{\Gamma,x:\sigma,\Gamma'\vdash}
\]

\[
\frac{\Gamma,\Gamma'\vdash M:\tau}{\Gamma,x:\sigma,\Gamma'\vdash M:\tau}
\]

\[
\frac{\Gamma,\Gamma'\vdash M\Rrightarrow M':\sigma}{\Gamma,x:\sigma,\Gamma'\vdash M\Rrightarrow M':\sigma}
\]

\[
\frac{\Gamma,\Gamma'\vdash M\Rrightarrow_{\ast}M':\sigma}{\Gamma,x:\sigma,\Gamma'\vdash M\Rrightarrow_{\ast}M':\sigma}
\]

\[
\frac{\Gamma,\Gamma'\vdash M\equiv M':\tau}{\Gamma,x:\sigma,\Gamma'\vdash M\equiv M':\tau}
\]

by mutual induction on the derivations.

\subsubsection{$\Rrightarrow$ is reflexive}

The following rule is admissible:

\[
\frac{\Gamma\vdash M:\sigma}{\Gamma\vdash M\Rrightarrow\,M:\sigma}\,\textrm{\ensuremath{\Rrightarrow}-refl}
\]

by induction

\subsubsection{$\equiv$ is reflexive}

The following rule is admissible:

\[
\frac{\Gamma\vdash M:\sigma}{\Gamma\vdash M\equiv M:\sigma}\,\textrm{\ensuremath{\equiv}-refl}
\]

by $\,\textrm{\ensuremath{\Rrightarrow\ast}-refl}$

\subsubsection{Context substitution}

For any derivation of $\Gamma\vdash N:\tau$ the following rules are
admissible:

\[
\frac{\Gamma,x:\tau,\Gamma'\vdash}{\Gamma,\Gamma'\left[x\coloneqq N\right]\vdash}
\]

\[
\frac{\Gamma,x:\tau,\Gamma'\vdash M:\sigma}{\Gamma,\Gamma'\left[x\coloneqq N\right]\vdash M\left[x\coloneqq N\right]\,:\,\sigma\left[x\coloneqq N\right]}
\]

\[
\frac{\Gamma,x:\tau,\Gamma'\vdash M\Rrightarrow M':\sigma}{\Gamma,\Gamma'\left[x\coloneqq N\right]\vdash M\left[x\coloneqq N\right]\Rrightarrow M'\left[x\coloneqq N\right]:\sigma\left[x\coloneqq N\right]}
\]

\[
\frac{\Gamma,x:\tau,\Gamma'\vdash M\Rrightarrow_{\ast}M':\sigma}{\Gamma,\Gamma'\left[x\coloneqq N\right]\vdash M\left[x\coloneqq N\right]\Rrightarrow_{\ast}M'\left[x\coloneqq N\right]:\sigma}
\]

\[
\frac{\Gamma,x:\tau,\Gamma'\vdash M\equiv M':\sigma}{\Gamma,\Gamma'\left[x\coloneqq N\right]\vdash M\left[x\coloneqq N\right]\,\equiv\,M'\left[x\coloneqq N\right]:\sigma\left[x\coloneqq N\right]}
\]

by mutual induction on the derivations. Specifically, at every usage
of $x$ from the var rule in the original derivation, replace the
usage of the var rule with the derivation of $\Gamma\vdash N:\tau$
weakened to the context of $\Gamma,\Gamma'\left[x\coloneqq N\right]\vdash N:\tau$,
and apply $\textrm{\ensuremath{\Rrightarrow}-refl}$ , $\textrm{\ensuremath{\Rrightarrow_{\ast}}-refl}$
or $\textrm{\ensuremath{\equiv}-refl}$ when needed.

\subsection{Computation}

\subsubsection{$\Rrightarrow$ preserves type of source}

The following rules are admissible:

\[
\frac{\Gamma\vdash N\Rrightarrow N':\tau}{\Gamma\vdash N:\tau}
\]

by induction

\subsubsection{$\Rrightarrow$-substitution }

The following rule is admissible:

\[
\frac{\Gamma,x:\sigma,\Gamma'\vdash M\Rrightarrow M':\tau\quad\Gamma\vdash N\Rrightarrow N':\sigma}{\Gamma,\Gamma'\left[x\coloneqq N\right]\vdash M\left[x\coloneqq N\right]\Rrightarrow M'\left[x\coloneqq N'\right]:\tau\left[x\coloneqq N\right]}
\]

by induction on the $\Rrightarrow$ derivations

\subsubsection{$\Rrightarrow$ is confluent}

if $\Gamma\vdash M\Rrightarrow N:\sigma$ and $\Gamma\vdash M\Rrightarrow N':\sigma$
then there exists $P$ such that 

$\Gamma\vdash N\Rrightarrow P:\sigma$ and $\Gamma\vdash N'\Rrightarrow P:\sigma$

by standard techniques

\subsection{$\Rrightarrow_{\ast}$}

\subsubsection{$\Rrightarrow_{\ast}$ is transitive}

The following rule is admissible:

\[
\frac{\Gamma\vdash M\Rrightarrow_{\ast}\,M':\sigma\quad\Gamma\vdash M'\Rrightarrow_{\ast}M'':\sigma}{\Gamma\vdash M\Rrightarrow_{\ast}\,M':\sigma}\,\textrm{\ensuremath{\Rrightarrow\ast}-trans}
\]

by induction

\subsubsection{$\Rrightarrow$ preserves type in destination}

\[
\frac{\Gamma\vdash N\Rrightarrow N':\tau}{\Gamma\vdash N':\tau}
\]

By induction on the $\Rrightarrow$ derivation with the help of the
substitution lemma.
\begin{itemize}
\item $\Pi\textrm{-\ensuremath{\Rrightarrow}}$
\begin{itemize}
\item $M'\left[x\coloneqq N',f\coloneqq\left(\mathsf{fun}\,f:\left(x.\tau'\right).\,x:\sigma'.M'\right)\right]\,:\,\tau'\left[x\coloneqq N'\right]$
by the substitution lemma used on the inductive hypotheses
\item $\tau\left[x\coloneqq N\right]\Rrightarrow\tau'\left[x\coloneqq N'\right]$
by $\Rrightarrow$-substitution , so $\tau\left[x\coloneqq N\right]\equiv\tau'\left[x\coloneqq N'\right]$
\item by the conversion rule $M'\left[x\coloneqq N',f\coloneqq\left(\mathsf{fun}\,f:\left(x.\tau'\right).\,x:\sigma'.M'\right)\right]\,:\,\tau\left[x\coloneqq N\right]$
\end{itemize}
\item $\Pi\textrm{-E-\ensuremath{\Rrightarrow}}$
\begin{itemize}
\item $M'\,N'\,:\,\tau\left[x\coloneqq N'\right]$, by $\Rrightarrow$-substitution
and reflexivity, $\tau\left[x\coloneqq N\right]\Rrightarrow\tau\left[x\coloneqq N'\right]$
, so $\tau\left[x\coloneqq N\right]\equiv\tau\left[x\coloneqq N'\right]$
\item by the conversion rule $M'\,N'\,:\,\tau\left[x\coloneqq N\right]$
\end{itemize}
\item $\Pi\textrm{-I-\ensuremath{\Rrightarrow}}$
\begin{itemize}
\item $\mathsf{fun}\,f:\left(x.\tau'\right).\,x:\sigma'.M'\,:\,\varPi x:\sigma'.\tau'$,
$\varPi x:\sigma.\tau\Rrightarrow\varPi x:\sigma'.\tau'$ , so $\varPi x:\sigma.\tau\equiv\varPi x:\sigma'.\tau'$
\item by the conversion rule $\mathsf{fun}\,f:\left(x.\tau'\right).\,x:\sigma'.M'\,:\,\varPi x:\sigma.\tau$
\end{itemize}
\item all other cases are trivial
\end{itemize}

\subsubsection{$\Rrightarrow_{\ast}$ preserves type}

The following rule is admissible:

\[
\frac{\Gamma\vdash M\Rrightarrow_{\ast}\,M':\sigma}{\Gamma\vdash M:\sigma}
\]

by induction

\[
\frac{\Gamma\vdash M\Rrightarrow_{\ast}\,M':\sigma}{\Gamma\vdash M':\sigma}
\]

by induction

\subsubsection{$\Rrightarrow_{\ast}$ is confluent}

if $\Gamma\vdash M\Rrightarrow_{\ast}\,N:\sigma$ and $\Gamma\vdash M\Rrightarrow_{\ast}\,N':\sigma$
then there exists $P$ such that 

$\Gamma\vdash N\Rrightarrow_{\ast}\,P:\sigma$ and $\Gamma\vdash N'\Rrightarrow_{\ast}\,P:\sigma$

Follows from $\textrm{\ensuremath{\Rrightarrow\ast}-trans}$ and the
confluence of $\Rrightarrow$ using standard techniques

\subsubsection{$\equiv$ is symmetric}

The following rule is admissible:

\[
\frac{\Gamma\vdash M\equiv N:\sigma}{\Gamma\vdash N\equiv M:\sigma}\,\textrm{\ensuremath{\equiv}-sym}
\]

trivial

\subsubsection{$\equiv$ is transitive}

\[
\frac{\Gamma\vdash M\equiv N:\sigma\qquad\Gamma\vdash N\equiv P:\sigma}{\Gamma\vdash M\equiv P:\sigma}\,\textrm{\ensuremath{\equiv}-trans}
\]

by the confluence of $\Rrightarrow_{\ast}$

\subsubsection{$\equiv$ preserves type}

The following rules are admissible:

\[
\frac{\Gamma\vdash M\equiv\,M':\sigma}{\Gamma\vdash M:\sigma}
\]

\[
\frac{\Gamma\vdash M\equiv\,M':\sigma}{\Gamma\vdash M':\sigma}
\]

by the def of $\Rrightarrow_{\ast}$

\subsubsection{Regularity}

The following rule is admissible:

\[
\frac{\Gamma\vdash M:\sigma}{\Gamma\vdash\sigma:\star}
\]

by induction with $\equiv$-preservation for the Conv case

\subsubsection{$\rightsquigarrow$ implies \textmd{$\Rrightarrow$} }

For any derivations of $\Gamma\vdash M:\sigma$, $M\rightsquigarrow M'$

\[
\Gamma\vdash M\Rrightarrow M':\sigma
\]

by induction on $\rightsquigarrow$

\subsubsection{$\rightsquigarrow$ preserves type}

For any derivations of $\Gamma\vdash M:\sigma$, $M\rightsquigarrow M'$,

\[
\Gamma\vdash M':\sigma
\]

since $\rightsquigarrow$ implies $\Rrightarrow$ and $\Rrightarrow$
preserves types

\subsection{Type constructors}

\subsubsection{Type constructors are stable}
\begin{itemize}
\item if $\Gamma\vdash\ast\Rrightarrow M:\sigma$ then $M$ is $\ast$
\item if $\Gamma\vdash\ast\Rrightarrow_{\ast}M:\sigma$ then $M$ is $\ast$
\item if $\Gamma\vdash\Pi x:\sigma.\tau\Rrightarrow M:\sigma$ then $M$
is $\Pi x:\sigma'.\tau'$ for some $\sigma',\tau'$
\item if $\Gamma\vdash\Pi x:\sigma.\tau\Rrightarrow_{\ast}M:\sigma$ then
$M$ is $\Pi x:\sigma'.\tau'$ for some $\sigma',\tau'$
\end{itemize}
by induction on the respective relations

\subsubsection{Type constructors definitionaly unique}

There is no derivation of $\Gamma\vdash\ast\equiv\Pi x:\sigma.\tau\,:\sigma'$
for any $\Gamma,\sigma,\tau,\sigma'$

from $\equiv\textrm{-Def}$ and constructor stability

\subsection{Canonical forms}

If $\lozenge\vdash v:\sigma$ then 
\begin{itemize}
\item if $\sigma$ is $\star$ then $v$ is $\star$ or $\Pi x:\sigma.\tau$
\item if $\sigma$ is $\Pi x:\sigma'.\tau$ for some $\sigma'$, $\tau$
then $v$ is $\mathsf{fun}\,f:\left(x.\tau'\right).\,x:\sigma''.P'$
for some $\tau'$, $\sigma''$, $P'$
\end{itemize}
By induction on the typing derivation
\begin{itemize}
\item Conv,
\begin{itemize}
\item if $\sigma$ is $\star$ then eventually, it was typed with type-in-type,
or $\Pi\textrm{-F}$. it could not have been typed by $\Pi\textrm{-I}$
since constructors are definitionaly unique
\item if $\sigma$ is $\Pi x:\sigma'.\tau$ then eventually, it was typed
with $\Pi\textrm{-I}$. it could not have been typed by type-in-type,
or $\Pi\textrm{-F}$ since constructors are definitionaly unique
\end{itemize}
\item type-in-type, $\lozenge\vdash v:\sigma$  is $\lozenge\vdash\star:\star$ 
\item $\Pi\textrm{-F}$, $\lozenge\vdash v:\sigma$  is $\lozenge\vdash\Pi x:\sigma.\tau\,:\star$ 
\item $\Pi\textrm{-I}$, $\lozenge\vdash v:\sigma$  is $\lozenge\vdash\mathsf{fun}\,f:\left(x.\tau\right).\,x:\sigma.M\,:\,\varPi x:\sigma.\tau$ 
\item no other typing rules are applicable
\end{itemize}

\subsection{Progress}

$\lozenge\vdash M:\sigma$ implies that $M$ is a value or there exists
$N$ such that $M\rightsquigarrow N$.

By direct induction on the typing derivation with the help of the
canonical forms lemma

Explicitly:
\begin{itemize}
\item $M$ is typed by the conversion rule, then by \textbf{induction},
$M$ is a value or there exists $N$ such that $M\rightsquigarrow N$ 
\item $M$ cannot be typed by the variable rule in the empty context
\item $M$ is typed by type-in-type. $M$ is $\star$, a value
\item $M$ is typed by $\Pi\textrm{-F}$. $M$ is $\Pi x:\sigma.\tau$,
a value
\item $M$ is typed by $\Pi\textrm{-I}$. $M$ is $\mathsf{fun}\,f:\left(x.\tau\right).\,x:\sigma.M'$,
a value
\item $M$ is typed by $\Pi\textrm{-E}$. $M$ is $P\,N$ then exist some
$\sigma,\tau$ for $\lozenge\vdash P:\Pi x:\sigma.\tau$ and $\lozenge\vdash N:\sigma$.
By \textbf{induction} (on the $P$ branch of the derivation) $P$
is a value or there exists $P'$ such that $P\rightsquigarrow P'$.
By \textbf{induction} (on the $N$ branch of the derivation) $N$
is a value or there exists $N'$ such that $N\rightsquigarrow N'$
\begin{itemize}
\item if $P$ is a value then by \textbf{canonical forms}, $P$ is$\mathsf{fun}\,f:\left(x.\tau\right).\,x:\sigma.P'$
and 
\begin{itemize}
\item if $N$ is a value then the one step reduction is $\left(\mathsf{fun}\,f:\left(x.\tau\right).\,x:\sigma.P'\right)\,N\,\rightsquigarrow\,P'\left[x\coloneqq N,f\coloneqq\mathsf{fun}\,f:\left(x.\tau\right).\,x:\sigma.M\right]$ 
\item otherwise there exists $N'$ such that $N\rightsquigarrow N'$, and
the one step reduction is $\left(\mathsf{fun}\,f:\left(x.\tau\right).\,x:\sigma.P'\right)\,N\,\rightsquigarrow\,\left(\mathsf{fun}\,f:\left(x.\tau\right).\,x:\sigma.P'\right)\,N'$ 
\end{itemize}
\item otherwise, there exists $P'$ such that $P\rightsquigarrow P'$ and
the one step reduction is $P\,N\rightsquigarrow P'\,N$
\end{itemize}
\end{itemize}

\subsection{Type Soundness}

For any well typed term in an empty context, no sequence of small
step reductions will cause result in a computation to ``get stuck''.
Either a final value will be reached or further reductions can be
taken. This follows by iterating the progress and preservation lemmas.

\section{Non-Properties}
\begin{itemize}
\item decidable type checking
\item normalization/logical soundness
\end{itemize}

\subsection*{Definitional Equality does not preserve type constructors on the
nose}

If $\Gamma\vdash\sigma\equiv\sigma':\star$ then 

if $\sigma$ is $\Pi x:\sigma''.\tau$ for some $\sigma''$, $\tau$
then $\sigma'$ is $\Pi x:\sigma'''.\tau'$ for some $\sigma'''$,
$\tau'$ 

counter example $\vdash\Pi x:\star.\star\equiv(\lambda x:\star.x)(\Pi x:\star.\star):\star$

this implies the additional work in the Canonical forms lemma

\section{differences from implementation}

differences from Agda development
\begin{itemize}
\item in both presentations standard properties of variables binding and
substitution are assumed
\item In Agda the parallel reduction relation does not track the original
typing judgment. This should not matter for the proof of confluence.
\item only proved the function part of the canonical forms lemma (all that
is needed for the proof)
\end{itemize}
differences from prototype
\begin{itemize}
\item bidirectional, type annotations are not always needed on functions
\item toplevel recursion in addition to function recursion
\item type annotations are not relevant for definitional equality
\end{itemize}

\section{Proof improvements}
\begin{itemize}
\item Clarify what unsoundness means 
\item proof outline at the top of document
\item better function notation
\item transition away from Greek letters
\end{itemize}
\bibliographystyle{plain}
\bibliography{C:/icfp/dtest/extended-abstract/dtest}

\end{document}
