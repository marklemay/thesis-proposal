%% LyX 2.3.0 created this file.  For more info, see http://www.lyx.org/.
%% Do not edit unless you really know what you are doing.
\documentclass[english,11pt]{article}
\usepackage[T1]{fontenc}
\usepackage[latin9]{inputenc}
\usepackage{amsmath}
\usepackage{amssymb}
\usepackage{cancel}

\makeatletter
%%%%%%%%%%%%%%%%%%%%%%%%%%%%%% User specified LaTeX commands.
\usepackage{typesetting/latex8}
\usepackage{times}
\usepackage{color}
\usepackage{epsfig}
\usepackage{graphicx}
\usepackage{graphics}
\usepackage{amsmath, nicefrac}
\usepackage{amssymb, amsthm}
\usepackage{wrapfig}
\usepackage{algorithm, algorithmic}
\usepackage{setspace}
\usepackage{caption}
\usepackage{float}
\usepackage{afterpage}
\usepackage{typesetting/abstract}
\usepackage{tabularx}
\usepackage{booktabs}
\usepackage{calc}
\usepackage{multirow}
\usepackage{longtable}
\usepackage{footnote}
\usepackage{threeparttable}
\usepackage{colortbl}
% \usepackage{tweaklist}
\usepackage{fancyhdr}
\usepackage[retainorgcmds]{IEEEtrantools}
\usepackage{floatflt}
\usepackage{xspace}

\usepackage{endnotes}
\usepackage{paralist}
\usepackage{typesetting/shortcuts}
\usepackage{tabulary}
\usepackage{mdwlist}
\usepackage{listings}
\usepackage{balance}
\usepackage{url}
\usepackage{parskip}
\usepackage{textcomp}
\usepackage{subcaption}

\usepackage{epstopdf}
\usepackage{fancyvrb}
% \usepackage[T1]{fontenc}

%-------------------------------------------------------------------------

%double spacing for document

\renewcommand{\textfraction}{0.1}
\renewcommand{\topfraction}{0.95}
\renewcommand{\bottomfraction}{0.95}
\renewcommand\floatpagefraction{0.9}
\setcounter{totalnumber}{50} \setcounter{topnumber}{50} \setcounter{bottomnumber}{50}
\renewcommand{\floatsep}{10pt}
\renewcommand{\intextsep}{10pt}
\setlength{\textfloatsep}{10pt}

\renewcommand{\headrulewidth}{0pt} \renewcommand{\footrulewidth}{0pt}

\newcommand{\eqnlinespace}{\\[5pt]}
\newcommand{\eqnlinespacelarge}{\\[10pt]}
\newcommand{\captionlinespace}{\\[0.05in]}

\renewcommand{\baselinestretch}{1.6}

%\renewcommand{\textwidth}{5.95in}
\setlength{\textwidth}{6.875in}

\renewcommand{\oddsidemargin}{0.5in}

%%%%%%%%%%%%%%%%%%%%%%%%%%%%%%%%%%%%%%%%%%%%

\makeatother

\usepackage{babel}
\begin{document}

\title{{\em Dissertation Prospectus}\\  A Full-Spectrum Dependently Typed
Language for Testing with Dynamic Equality}

\author{Mark Lemay\\ Department of Computer Science, Boston University \\ \today}
\maketitle
\begin{abstract}
Dependent type systems are a powerful tool to eliminate bugs from
programs. Unlike other systems of formal methods, dependent types
systems can re-use the methodology and syntax that functional programmers
are already familiar with for the construction of formal proofs. This
insight has lead to several full-spectrum languages that try to present
programmers with a consistent view of proofs and programs. However,
these languages still have substantial usability issues: missing features
like general recursion, confusingly conservative equality, and no
straightforward way to test and use properties that have not yet been
proven.

These issues are not superficial, but are tied to some of the conventional
assumptions of dependent type theory. Logical soundness is essential
when dependent types are considered as a mathematical foundation,
but may be too restrictive in a programming language. Conservative
equality is easy to hand-wave away informally, but it makes many programs
difficult to type check. Users often experience these language properties
as inexplicable static errors, and cannot debug their programs using
conventional dynamic debugging. In practice exploratory programming
is difficult in these systems.

A possible solution is to design a dependently typed language with
programmatic features that supports type sound, but logically unsound
execution. Programmers can trade poor static errors for precise counterexamples
that are made available at runtime. Progress towards that solution
has been made by developing a language with dependent types and programming
oriented features, characterizing some correctness criteria for blame
tracking, and implementing a subset of that language that matches
those criteria.% awk
\end{abstract}
\tableofcontents{}

\section{Introduction}

The Curry-Howard correspondence identifies functions with theorems,
providing a mutually beneficial link between well explored areas of
math and software engineering. This connection is most pronounced
in dependent typed systems that provide a common language to write
programs about proofs and proofs about programs. Specifically, dependent
type systems allow types to depend on terms, supporting an extremely
powerful system of specification, computation and proof evidence. 

For instance, in a dependently typed language it is possible to prove
the correctness of a sorting function 

\[
sort\,:\,\left(input:List\,\mathbb{N}\right)\rightarrow\Sigma ls:List\,\mathbb{N}.IsSorted\,input\,ls
\]

by providing an appropriate term of that type. Unlike other systems
of formal methods, the additional logical power does not require the
programmer understand any additional syntax or semantics. From the
programmer's perspective the function arrow and the implication arrow
are the same. The proof $IsSorted$ is no different then any other
datatype like $List$ or $\mathbb{N}$. 

The promise of dependent types in a practical programming language
has inspired research projects for decades. There have been many formalization
and prototypes that make different compromises in the design space.
One of the most popular styles is ``Full-spectrum'' dependent types,
these languages tend to have a minimalist approach: computation can
appear anywhere in a term or type. Such a design purposely exposes
the Curry-Howard correspondence, as opposed to merely using it as
a convenient logical foundation: a proof has the exact same syntax
and behavior as a program. Even though this style makes writing efficient
programs hard, and drastically complicates the ability to encode effects,
it can be seen in some of the most popular dependently typed languages
such as Agda and Idris.

Despite the potential, users often find these systems difficult to
use. The common symptom of these issues can be seen in the confusing
error messages these languages produce. For instance in Agda this
reasonable looking program 

\begin{align*}
\mathtt{Vec} & :*\rightarrow\mathbb{N}\rightarrow*\\
\mathtt{rep} & :\left(x:\mathbb{N}\right)\rightarrow\mathtt{Vec\,\mathbb{B}}\,x\\
\mathtt{head} & :\left(x:\mathbb{N}\right)\rightarrow\mathtt{Vec}\,\mathbb{B}\,\left(1+x\right)\rightarrow\mathbb{B}\\
\cancel{\vdash} & \lambda x.\mathtt{head}\,x\,\left(\mathtt{rep}\,\left(x+1\right)\right)\,:\,\mathbb{N}\rightarrow\mathbb{B}
\end{align*}

will give the error ``x + 1 != suc x of type $\mathbb{N}$ when checking
that the expression rep (x + 1) has type Vec $\mathbb{B}$ (1 + x)''.
The error is confusing since it objects to an intended property of
addition, and if addition were buggy no hints are given to fix the
problem. While an expert in Type Theory can appreciate the subtleties
of definitional equality, programmers would prefer an error message
that gives a specific instance of $x$ where $x+1\neq1+x$ or be allowed
to run their program. 

Strengthening the equality relation in dependently typed languages
has motivated many research projects (to name a few \cite{cockx2021taming,sjoberg2015programming}).
However, it is unlikely those impressive efforts are suitable for
non-exerts, since programmers expect the data types and functions
they define to have the properties they were intended to have. None
of these projects makes the underling equality less complicated. No
system will be able to verify every ``obvious'' equality for arbitrary
user defined data types and functions statically, since program equivalence
is famously undecidable.

Alternatively we could assume the equalities hold and discover a concrete
inequality as a runtime error. There is some evidence that specific
examples like this can help clarify the error messages in OCaml\cite{10.1145/2951913.2951915}
and there has been an effort to make refinement type error messages
more concrete and other systems like Liquid Haskell\cite{10.1145/3314221.3314618}.
This leads to a different workflow then traditional type systems,
instead of verifying everything first and only then executing the
program, execution and type checking can both inform the programmer.% what is "this"?

Several steps have been taken to make this new workflow possible.
Section 2 will describe a conventional dependently typed base language
with some programmatic features. Section 3 describes dynamic dependent
equality and the conditions that would make an implementation reasonable.
Section 4 covers some of the other features a language in this style
could support. % progress has been separated into several independent results

\section{A Dependently Typed Base Language}

The dynamic dependent equality of the next section is hard to study
without a more conventional dependently typed language to serve as
a reference. This base language contains the key features that should
be supported: user defined dependent functions and dependent datatypes.
The base language uses a ``full-spectrum'' style. The dynamic language
from the next section can be elaborated from and compared to this
concrete implementation.

The language is pure in the sense of Haskell, supporting only non-termination
and unchecked errors as effects. Combining other effects with full-spectrum
dependent types is substantially more difficult because effectful
equality is hard to characterize for individual effects and especially
hard for effects in combination. Several attempts have been made to
combine dependent types with more effects, \cite{pedrot2020fire}
\cite{ahman2017handling,ahman2017fibred}\cite{pedrot2020fire} but
there is still a lot of work to be done. Effects, though undoubtedly
useful, are not considered in the base language.

Since this work emphasizes programming over theorem proving, the language
contains these logically dubious features:
\begin{itemize}
\item Unrestricted recursion (no required termination checking)
\item Unrestricted user defined dependent data types (no requirement of
strict positivity)
\item Type-in-Type (no predictive hierarchy of universes)
\end{itemize}
Any one of these features can result in logical unsoundness\footnote{Every type is inhabited by an infinite loop.},
but they seem helpful for mainstream functional programming. In spite
of logical unsoundness, the resulting language still has type soundness\footnote{No term with a reduct that applies an argument to a non-function in
the empty context will type.}. This seems suitable for a programming language since logically sound
proofs can still be encoded and logical unsoundness can be discovered
through traditional testing, or warned about in a non-blocking way.
Importantly, no desirable computation is prevented. % what is "this"?

Logical soundness seems not to matter in programming practice. For
instance, in ML the type $\mathtt{f:Int->Int}$ does not imply the
termination of $\mathtt{f\,2}$. While unproductive non-termination
is always a bug, it seems an easy bug to detect and fix. In mainstream
languages, types help to communicate intent, beyond what is strictly
guaranteed by the type system.

The base language still supports a partial correctness property for
first order data types when run with Call-by-Value. For instance:

\[
\vdash M:\,\sum x:\mathbb{N}.\mathtt{IsEven}\,x
\]

$\mathtt{fst}\,M$ may not terminate, but if it does, $\mathtt{fst}\,M$
will be an even $\mathbb{N}$. However, this property does not extend
to functions

\[
\vdash M:\,\sum x:\mathbb{N}.\left(y:\mathbb{N}\right)\rightarrow x\leq y
\]

it is possible that $\mathtt{fst}\,M\equiv7$ if 

\[
M\equiv\left\langle 7,\lambda y.\mathtt{loopForever}\right\rangle 
\]


\subsection{Prior work}

While many of these features have been explored in theory and implemented
in practice, I am unaware of any development with exactly this formulation.

Unsound logical systems that are acceptable programming languages
go back to at least Church's lambda calculus which was originally
intended to be a logical foundation for mathematics. In the 1970s,
Martin Lof proposed a system with Type-in-Type that was shown logically
unsound by Girard (as described in the introduction in \cite{Martin-Lof-1972}).
In the 1980s, Cardelli explored the domain semantics of a system with
general recursive dependent functions and Type-in-Type\cite{cardelli1986polymorphic}.

The first proof of type soundness for a language with general recursive
dependent functions, Type-in-Type, and dependent data that I am aware
of came form the Trellys Project \cite{sjoberg2012irrelevance}. At
the time their language had several additional features, not included
in the base language. Additionally the base language uses a simpler
notion of equality and dependent data resulting in an arguably simpler
proof of type soundness. Later work in the Trellys Project\cite{casinghino2014combining,casinghino2014combiningthesis}
used modalities to separate the terminating and non terminating fragments
of the language, though the annotation burden seems too high in practice.
In general, the base language has been deeply informed by the Trellys
Project\cite{kimmell2012equational}\cite{sjoberg2012irrelevance}\cite{casinghino2014combining,casinghino2014combiningthesis}
\cite{sjoberg2015programming} \cite{sjoberg2015dependently} and
the Zombie Language\footnote{https://github.com/sweirich/trellys}
it produced.

Several implementations support this combination of features without
proofs of type soundness. Cayenne \cite{10.1145/289423.289451} was
an early Haskell like language that combined dependent types with
and data and non-termination. Agda supports general recursion and
Type-in-Type with compiler flags, and can simulate some non-positive
data types using coinduction. Idris supports similar ``unsafe''
features.

\cite{jia2010dependent} introduces a similar ``partial correctness''
criterion for dependent languages with non-termination.

\section{A Language with Dynamic Type Equality}

A key issue with dependent type theories is the characterization of
definitional equality. Since computation can appear at the type level,
and types must be checked for equality, dependent type theories must
define what computational equalities they intend to equate. For instance,
the base language follows the common choice of $\alpha\beta$ equivalence
of terms. However, this causes many programs to not type-check:

$\lambda x.\mathtt{head}\,x\,\left(\mathtt{rep}\,\left(x+1\right)\right)\,:\,\mathbb{N}\rightarrow\mathbb{B}$
Since $1+x$ does not have the same definition as $x+1$.

This is a widely recognized issue with dependent type theories. However,
most attempts to improve the equality relation intend to preserve
decidable type checking and/or logical soundness, so equality will
never be complete\footnote{I am also unaware of any suitable notion of complete extensional equality
for dependent type theory though it is considered in \cite{sjoberg2015dependently}
.}. Since dependently typed languages with the practical features outlined
in base language are already incompatible with logical soundness and
decidable type checking, these concerns no longer apply.

The base language can be extended to a cast language that supports
the expectation of the original typing. Many programs that do not
type in the base language can be elaborated into the cast language.
The cast language has a weaker notion of type soundness such that 
\begin{enumerate}
\item $\vdash_{c}e':M'$ then
\begin{enumerate}
\item $e'\downarrow v'$ and $\vdash_{c}v':M'$ 
\item or $e'\uparrow$ 
\item or $e'\downarrow blame$ 
\end{enumerate}
\end{enumerate}
Type soundness is preserved, or inequality can be proven at a specific
source location. In the example above $\lambda x.\mathtt{head}\,x\,\left(\mathtt{rep}\,\left(x+1\right)\right)\,:\,\mathbb{N}\rightarrow\mathbb{B}$
will not emit any errors at compile time or runtime (though a static
warning may be given). 

If the example is changed to

\[
\lambda x.\mathtt{head}\,x\,\left(\mathtt{rep}\,x\right)\,:\,\mathbb{N}\rightarrow\mathbb{B}
\]

and the function is called at runtime, the blame tracking system will
blame the exact static location that uses unequal types with a direct
proof of inequality, allowing an error like ``failed at application 

$\left(\mathtt{head}\,x:\mathtt{Vec}\,\underline{\left(1+x\right)}\,\mathbb{B}\rightarrow...\right)\left(rep\,x:\mathtt{Vec}\,\underline{x}\,\mathbb{B}\right)$
since when $x=3$, $1+x=4\neq3=x$'', regardless of when the function
was called in the program and where the discrepancy was discovered.
This improves on the naive solution of completely ignoring type annotations
for execution, since without type soundness, the returned error may
appear unrelated to the problematic type assumption (``3 applied
to 7 but 7 is not a function''). 

Just as standard type theories allow many possible characterizations
of equality that support logical soundness, there are many choices
of runtime checking that support this weaker notion of type soundness.
The minimal choice is likely too permissive in practice: in the example
above, it would only flag an error when the function is applied to
0. Alternatively, I conjecture that runtime-checking that matches
the partial correctness criteria above would be reasonably intuitive.
Extending checks into non-dependent function types also seems reasonable,
and would allow simple types to be completely checked.

Taking inspiration from the ``gradual guarantee'' of gradual typing,
there are several basic properties in addition to weakened type soundness
that this cast language hopes to fulfill:
\begin{enumerate}
\item $\vdash e:M$, $elab\left(M,*\right)=M'$, and $elab\left(e,M'\right)=e'$
then $\vdash_{c}e':M'$.
\item $\vdash_{c}e':M'$ and $e'\downarrow blame$ then there is no $\vdash e:M$
such that $elab\left(M,*\right)=M'$, $elab\left(e,M'\right)=e'$
\item $\vdash_{c}e':*$ and $elab\left(e,*\right)=e'$ then
\begin{enumerate}
\item if $e'\downarrow*$ then $e\downarrow*$
\item if $e'\downarrow(x:M')\rightarrow N'$ then $e\downarrow(x:M)\rightarrow N$ 
\item if $e'\downarrow TCon\,\overline{M'}$ then $e\downarrow TCon\,\overline{M}$ 
\end{enumerate}
\end{enumerate}
The first condition states that every typed term in the base language
can be embedded in the cast language. The second condition shows that
errors are not spurious. The third condition shows that except for
error, observations are consistent with the base language (with large
eliminations, term constructor observations are also consistent).

\subsection{Prior work}

It is unsurprising that dynamic equality shares many of the same concerns
as the large amount of work for contracts, hybrid types, gradual types,
and blame. In fact, this work could be seen as gradualizing the Reflection
Rule in Extensional Type Theory.% cite ETT?

Blame has been strongly advocated for in \cite{10.1007/978-3-642-00590-9_1,wadler:LIPIcs:2015:5033}.
Blame tracking can help establish the reasonableness of monitoring
systems by linking a dynamic failure directly to the broken static
invariant. Blame is also a key ingredient of good error messages.
However, as many authors have noticed, proving blame correctness is
tedious and error prone, it is often only conjectured.

The basic correctness conditions are inspired by the Gradual Guarantee
\cite{siek_et_al:LIPIcs:2015:5031}. The implementation also takes
inspiration from ``Abstracting gradual typing''\cite{10.1145/2837614.2837670},
where static evidence annotations become runtime checks. Unlike some
impressive attempts to gradualize the polymorphic lambda calculus
\cite{10.1145/3110283}, dynamic equality does not attempt to preserve
any parametric properties of the base language. It is unclear if such
a restriction to parametric properties would be desirable for a dependently
typed language.

A direct attempt has been made to gradualize a full spectrum dependently
typed language to an untyped lambda calculus using the AGT philosophy
in \cite{10.1145/3341692}. However that system retains the intentional
style of equality and user defined data types are not supported. The
paper is largely concerned with establishing decidable type checking
via an approximate term normalization.

A refinement type system with higher order features is gradualized
in \cite{c4be73a0daf74c9aa4d13483a2c4dd0e} though it does not appear
powerful enough to be characterized as a full-spectrum dependent type
theory. \cite{c4be73a0daf74c9aa4d13483a2c4dd0e} builds on earlier
refinement type system work, which described itself as ``dynamic''.
A notable example is \cite{10.1007/1-4020-8141-3_34} which describes
a refinement system that limits predicates to base types.

\section{Further Features}

Dynamic equality appears to be a prerequisite to other interesting
possibilities.

\subsection{Prototyping proofs and programs}

Just as ``obvious'' equalities are missing from the definitional
relation, ``obvious'' proofs and programs are not always conveniently
available to the programmer. For instance, in Agda it is possible
to write a sorting function quickly using simple types. With effort
is it possible to prove that sorting procedure correct by rewriting
it with the necessarily invariants. However, very little is offered
in between. The problem is magnified if module boundaries hide the
implementation details of a function, since the details are exactly
what is needed to make a proof! This is especially important for larger
scale software where a library may require proof terms that while
``correct'' are not constructable from the exports of other libraries.

The solution proposed here is some additional syntax that will search
for a term of the type when resolved at runtime. Given the sorting
function 

\[
\mathtt{sort}:\mathtt{List}\,\mathbb{N}\rightarrow\mathtt{List}\,\mathbb{N}
\]

and given the first order predicate that 

\[
\mathtt{IsSorted}:\mathtt{List}\,\mathbb{N}\rightarrow*
\]

then it is possible to assert that $\mathtt{sort}$ behaves as expected
with

\[
\lambda x.?:\left(x:\mathtt{List}\,\mathbb{N}\right)\rightarrow\mathtt{IsSorted}\left(\mathtt{sort}x\right)
\]

This term will act like any other function at runtime, given a list
input it will verify that the $\mathtt{sort}$ function correctly
handles that input, gives an error, or non-terminates.

Additionally, this would allow simple prototyping form first order
specification. For instance,

\begin{align*}
data\,\mathtt{Mult} & :\mathbb{N}\rightarrow\mathbb{N}\rightarrow\mathbb{N}\rightarrow*\,where\\
\mathtt{base} & :\left(x:\mathbb{N}\right)\rightarrow\mathtt{Mult}0\,x\,0\\
\mathtt{suc} & :\left(x\,y\,z:\mathbb{N}\right)\rightarrow\mathtt{Mult}\,x\,y\,z\rightarrow\mathtt{Mult}\,\left(1+x\right)\,y\,(y+z)
\end{align*}

can be used to prototype

\[
\mathtt{div}=\lambda z.\lambda x.\mathtt{fst}\left(?:\sum y:\mathbb{N}.\mathtt{Mult}x\,y\,z\right)
\]

The term search can be surprisingly subtle. For instance,

\[
?:\sum f:\mathbb{N}\rightarrow\mathbb{N}.\mathtt{Id}\left(f,\lambda x.x+1\right)\&\mathtt{Id}\left(f,\lambda x.1+x\right)
\]

depends on the definitional properties of functions. To avoid this
subtly I plan to only support term search over first order data.

Though the proof search is currently primitive, better search methods
could be incorporated in future work.

\subsubsection{Prior work}

Proof search is often used for static term generation in dependently
typed languages (for instance Coq tactics). A first order theorem
prover is attached to Agda in \cite{norell2007towards}.

Twelf made use of runtime proof search but the underling theory cannot
be considered full-spectrum.

\subsection{Testing dependent programs}

Both dynamic equalities and dynamic proof search vastly weaken the
guarantees of normal dependent type systems. Programmers still would
like evidence of correctness, even while they intend to provide full
proofs in the future. However, there are few options available in
full-spectrum dependently typed languages aside from costly and sometimes
unconstructable proofs.

The mainstream software industry has similar needs for evidence of
correctness, and has made use of testing done in a separate execution
phase. Given the rich specifications that dependent types provide
it is possible to improve on the hand crafted tests used by most of
the industry. Instead we can use a type directed symbolic execution,
to run questionable equalities over concrete values and precompute
the searches of proof terms. Precomputed proof terms can be cached,
so that exploration is not too inefficient in the common case of repeating
tests at regular intervals of code that is mostly the same. Precomputed
terms can be made available at runtime, excusing the inefficient search
procedure. 

Interestingly dynamic equality is necessary for testing like this,
since otherwise, definitional properties of functions would need to
be accounted for. Using dynamic equality it is possible only consider
the extensional behavior of functions.

Finally, future work can add more advanced methods of testing and
proof generation. This architecture should make it easier to change
exploration and search procedures without changing the underlining
definitional behavior.

\subsubsection{Prior work }

\subsubsection{Symbolic Execution}

Most research for Symbolic Execution targets popular imparative languages
(like C) and uses SMT solvers to efficiently explore conditional branches
that depend on base types. Most work does not support higher order
functions or makes simplifying assumptions about the type system.
There are however some relevant papers:
\begin{itemize}
\item \cite{10.1145/3314221.3314618} presents a symbolic execution engine
supporting Haskell's lazy execution and type system. Higher order
functions are not handled
\item The draft work\cite{2006.11639}, handles higher order functions as
inputs and provides a proof of completeness
\item Symbolic execution for higher order functions for a limited untyped
variant of PCF is described in \cite{nguyen2017higher}
\end{itemize}

\subsubsection{Testing dependent types}

There has been a long recognized need for testing in addition to proving
in dependent type systems
\begin{itemize}
\item In \cite{dybjer2003combining} a QuickCheck style framework was added
to an earlier version of Agda
\item QuickChick\footnote{https://github.com/QuickChick/QuickChick} \cite{denes2014quickchick}\cite{lampropoulos2017generating,lampropoulos2017beginner,lampropoulos2018random}
is a research project to add testing to Coq. However, testing requires
building types classes that establish the properties needed by the
testing framework such as decidable equality. This is presumably out
of reach of novice Coq users.
\end{itemize}

\section{Status and Plan}

\subsection{Status}

I am rewriting a prototype for dynamic equality to support for all
the features of the base language. The fragment for dependent function
types and type universes satisfies all the conditions above. I believe
that all conditions can be made to hold with data types as well.

The proof search prototype written for the extended abstract presented
last summer, will need to be corrected to account for dynamic equality.

\subsection{Plan}
\begin{itemize}
\item February: Write up proofs for dynamic equality 
\item March: Submit to ICFP 2021 
\item April: Finish re-writing the proptotype for dynamic equality
\item May: Rewrite the proof search and testing prototype
\item June: Draft thesis
\item July: Defend
\end{itemize}
\bibliographystyle{plain}
\bibliography{C:/icfp/dtest/extended-abstract/dtest}

\end{document}
