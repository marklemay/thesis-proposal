%% LyX 2.3.0 created this file.  For more info, see http://www.lyx.org/.
%% Do not edit unless you really know what you are doing.
\documentclass[english,11pt]{article}
\usepackage[T1]{fontenc}
\usepackage[latin9]{inputenc}
\usepackage{amsmath}
\usepackage{amssymb}
\usepackage{cancel}

\makeatletter
%%%%%%%%%%%%%%%%%%%%%%%%%%%%%% User specified LaTeX commands.
\usepackage{typesetting/latex8}
\usepackage{times}
\usepackage{color}
\usepackage{epsfig}
\usepackage{graphicx}
\usepackage{graphics}
\usepackage{amsmath, nicefrac}
\usepackage{amssymb, amsthm}
\usepackage{wrapfig}
\usepackage{algorithm, algorithmic}

\makeatother

\usepackage{babel}
\begin{document}

\title{\textbf{Dissertation Prospectus} A Full-Spectrum Dependently typed
language for testing with dynamic equality}

\author{Mark Lemay}
\maketitle
\begin{abstract}
Dependent type systems offer a powerful tool to eliminate bugs from
programs. Interest in dependent types is often driven by the inherent
usability of such systems: Dependent types systems can re-use that
methodology and syntax that functional programmers are familiar with
for formal proofs. This insight has lead to several Full-Spectrum
languages that try and present programmers with a consistent and unrestricted
view of proofs and programs. However these languages still have substantial
usability issues: missing features like general recursion, confusingly
conservative equality, an inability to prototype, and no strait forward
way to test specifications that have not yet been proven.

I attempt to solve these problems by building a new language that
contains standard functional programming features such as general
recursion, with a gradualized equality, runtime proof search and a
testing system.
\end{abstract}
\tableofcontents{}

\section{Introduction}

The promise of dependent types in a practical programming language
has been the goal of research projects for decades. There have been
many formalization and prototypes that make different compromises
in the design space. One popular direction to explore is ``Full-spectrum''
dependently types languages, these languages tend to have a minimalist
approach: computation can appear anywhere in a term or type. Such
a design purposely exposes the Curry-Howard correspondence, as opposed
to trying to hide it as a technical foundation: a proof has the exact
same syntax and behavior as a program. This direct approach tries
to make clear to the programmer the subtleties of the proof system
that are often obscured by other formal method systems. Even though
this style makes writing efficient programs hard, and drastically
complicates the ability to encode effects, it can be seen in some
of the most popular dependently typed languages (notably Agda and
Idris).

However there are several inconveniences with languages in this style: 
\begin{enumerate}
\item A restriction on standard programming features, such as general recursion
\item A subtle and weak notion of equality
\item difficulties in prototyping proofs and programs
\item Difficulties in testing programs that make use of dependent types
\end{enumerate}
While each problem will be treated as separately as possible, the
nature of dependent types requires that equality is modified before
testing and prototyping can be handled. The notion of equality itself
is also very sensitive to which programmatic features are included.
My thesis will solve these problems by
\begin{itemize}
\item Defining a full-spectrum dependently typed base language, with a few
of the most essential programming features like general recursion
and user defined data types
\item A cast language that supports dynamic equality checking
\item Syntax that supports runtime proof search
\item A symbolic testing system that will exercise terms with uncertain
equalities and runtime proof search
\end{itemize}

\section{A Dependently Typed Base Language}

The base language contains the features:
\begin{itemize}
\item Unrestricted dependent data types (no requirement of strict positivity)
\item Unrestricted recursion (no required termination checking)
\item Type-in-type (no predictive hierarchy of universes)
\end{itemize}
Any one of these features can result in logical unsoundness\footnote{Every type is inhabited by an infinite loop},
but they are widely used in mainstream functional programming. In
spite of the logical unsoundness, the resulting language is still
has type soundness\footnote{no term with a reduct that applies an argument to a non-function in
the empty context will type}. Type checking is undecidable for this language, however this has
not been a problem in practice\footnote{While languages like Coq and Agda claim decidable typechecking, it
is easy to construct terms who's type verification would exceed the
computational resources of the universe}. 

The type theory is intutionistic, definitional equality is the $\alpha\beta$
equivalence of terms. The implementation is written in a bidirectional
style allowing some annotations to be inferred.

Though this language is non-terminating it supports a partial correctness
property for first order data types when run with CBV, for instance:

\[
\vdash M:\sum x:\mathbb{N}.\mathtt{IsEven}\,x
\]

$\mathtt{fst}\,M$ may not terminate, but if it does, $\mathtt{fst}\,M$
will be an even $\mathbb{N}$. However, this property does not extend
to functions

\[
\vdash M:\sum x:\mathbb{N}.\left(y:\mathbb{N}\right)\rightarrow x\leq y
\]

it is possible that $\mathtt{fst}\,M\equiv7$ if 

\[
\left\langle 7,\lambda y.\mathtt{loopForever}\right\rangle 
\]

The hope would be that the type is sufficient to communicate intent,
in the same way unproductive non-termination is trappable in the vast
majority of typed languages but still considered a bug.

\subsection{Prior work for the Base Language}

While the base language is in line with prior research, I am unaware
of any development with exactly these features. Agda supports general
recursion and Typy-in-type with compiler flags, and can some non-positive
data types using coinduction. Idris supports similar ``unsafe''
features. Meta-theoretically, this base language is similar to \cite{sjoberg2012irrelevance}
though data and equality are formulated differently. The base language
has been deeply informed by the Trellys Project\cite{kimmell2012equational}\cite{sjoberg2012irrelevance}\cite{casinghino2014combining,casinghino2014combiningthesis}
\cite{sjoberg2015programming} \cite{sjoberg2015dependently} and
the Zombie Language\footnote{https://github.com/sweirich/trellys}
it produced.

\cite{jia2010dependent} claims a similar ``partial correctness''
criterion.

\section{A Language with Dynamic Dependent Equality}

A key issue with full-spectrum dependent type theories is the characterization
of definitional equality. Since computation can appear at the type
level, and types must be checked for equality, traditional dependent
type theories pick a subset of equivalences to support. For instance,
the base language follows the common choice of $\alpha\beta$ equivalence
of terms. However this causes many obvious programs to not type-check:

\begin{align*}
\mathtt{Vec} & :\mathbb{N}\rightarrow*\rightarrow*\\
\mathtt{rep} & :\left(x:\mathbb{N}\right)\rightarrow\mathtt{Vec}\,x\,\mathbb{B}\\
\mathtt{head} & :\left(x:\mathbb{N}\right)\rightarrow\mathtt{Vec}\,\left(1+x\right)\,\mathbb{B}\rightarrow\mathbb{B}\\
\cancel{\vdash} & \lambda x.\mathtt{head}\,x\,\left(\mathtt{rep}\,\left(x+1\right)\right)\,:\,\mathbb{N}\rightarrow\mathbb{B}
\end{align*}

Since $1+x$ does not have the same definition as $x+1$.

Overly fine definitional equalities directly results in the poor error
messages that are common for dependently typed languages \cite{eremondi2019framework}.
For instance, the above will give the error message ``x + 1 != suc
x of type \ensuremath{\mathbb{N}} when checking that the expression
rep (x + 1) has type Vec Bool (1 + x)'' in Agda. The error is confusing
since it objects to an obvious property of addition, and if addition
were buggy no hints would be given to fix the problem. Ideally the
error messages would give a specific instance of $x$ where $x+1\neq1+x$
or remain silent. There is some evidence that specific examples can
help clarify the error messages in OCaml\cite{10.1145/2951913.2951915}
and there has been an effort to make refinement type error messages
more concrete in Liquid Haskell\cite{10.1145/3314221.3314618}.

Strengthening the equality relation dependently typed languages is
used to motivate many research projects (to name a few \cite{cockx2021taming,sjoberg2015programming,HoTTbook}).
However, every formulation I am aware of intends to preserve decidable
type checking or logical soundness, so equality will never be complete\footnote{I am also unaware of any suitable notion of complete equality though
it is considered in \cite{sjoberg2015dependently} }. Since I intend to dispense with both decidable type checking or
logical soundness I can propose a language with an equality that is
more convenient in practice.

Building off the base language I purpose a dynamic cast language,
a cast type system and a partial elaboration function that satisfies
the following basic guarantees:
\begin{enumerate}
\item $\vdash e:M$ and $elab\left(e,M\right)=e'$ then $\vdash e':M'$
for some $\vdash M':*$ .
\item $\vdash e':M'$ and $e'\downarrow blame$ then there is no $\vdash e:M$
such that $elab\left(e,M\right)=e'$
\item $\vdash e':*$ and $elab\left(e,*\right)=e'$ then
\begin{enumerate}
\item if $e'\downarrow*$ then $e\downarrow*$
\item if $e'\downarrow(x:M')\rightarrow N'$ then $e\downarrow(x:M)\rightarrow N$ 
\item if $e'\downarrow TCon\triangle'$ then $e\downarrow TCon\triangle$ 
\end{enumerate}
\item $\vdash e':M'$ then
\begin{enumerate}
\item $e'\downarrow v'$ and $\vdash v':M'$ 
\item or $e'\uparrow$ 
\item or $e'\downarrow blame$ 
\end{enumerate}
\end{enumerate}
In the example above $\lambda x.\mathtt{head}\,x\,\left(\mathtt{rep}\,\left(x+1\right)\right)\,:\,\mathbb{N}\rightarrow\mathbb{B}$
will not emit any errors at compile time or runtime (though a warning
may be given). If the example is changed to

\[
\lambda x.\mathtt{head}\,x\,\left(\mathtt{rep}\,x\right)\,:\,\mathbb{N}\rightarrow\mathbb{B}
\]

no static error will be given, but if the function is called a runtime
error will be throughn. The blame tracking system will blame the exact
static location that uses unequal types with a direct proof of inequality,
allowing an error like ``failed at application $\left(\mathtt{head}\,x:\mathtt{Vec}\,\underline{\left(1+x\right)}\,\mathbb{B}\rightarrow...\right)\left(rep\,x:\mathtt{Vec}\,\underline{x}\,\mathbb{B}\right)$
since when $x=3$, $1+x=4\neq3=x$'', regardless of where in the
program the the function was called.

\subsection{Prior work}

It is unsurprising that dynamic quality is shares many of the same
concerns as the large amount of work contracts, hybrid types, gradual
types, and blame. In fact, this work could be seen as gradualizing
the Reflection Rule in Extensional Type Theory.

Blame has been strongly advocated for in \cite{10.1007/978-3-642-00590-9_1,wadler:LIPIcs:2015:5033}.
Blame tracking can establish the reasonableness of gradual typing
systems, though as many authors have noticed, proving blame correctness
is tedious and error prone, many authors only conjecture it for their
systems.

The basic correctness conditions are inspired by the Gradual Guarantee
\cite{siek_et_al:LIPIcs:2015:5031}. The implementation also takes
inspiration from ``Abstracting gradual typing''\cite{10.1145/2837614.2837670},
where static evidence annotations become runtime checks. Unlike some
impressive attempts to gradualize the polymorphic lambda calculus
\cite{10.1145/3110283}, dynamic equality does not attempt to preserve
any parametric properties of the base language.

A direct attempt has been made to gradualize a full spectrum dependently
typed language to an untyped lambda calculus using the AGT philosophy
in \cite{10.1145/3341692}. However that system retains the definitional
style of equality and user defined data types are not supported. The
paper is largely concerned with establishing decidable type checking
via an approximate term normalization.

A refinement type system with higher order features is gradualized
in \cite{c4be73a0daf74c9aa4d13483a2c4dd0e} though it does not appear
powerful enough to be characterized a a full-spectrum dependent type
theory. \cite{c4be73a0daf74c9aa4d13483a2c4dd0e} builds on earlier
refinement type system work, which described itself as ``dynamic''
. A notable example is \cite{10.1007/1-4020-8141-3_34} which describes
a refinement system that limit's predicates to base types.

\section{Prototyping proofs and programs}

Just as ``obvious'' equalities are missing from the definitional
relation, ``obvious'' proofs and programs are not always conveniently
available to the programmer. For instance, in Agda it is possible
to write a sorting sorting function quickly using simple types. With
expertise and effort is it possible to prove that sorting procedure
correct by rewriting it with the necessarily invariants. However very
little is offered in between. The problem is magnified if module boundaries
hide the implementation details of a function, since the details are
exactly what is needed to make a proof! This is especially important
for larger scale software where a library may expect proof terms that
while ``correct'' are not constructible from the exports of the
other library.

The solution proposed here is some additional syntax that will search
for a term of the type when resolved at runtime. Given the sorting
function 

\[
\mathtt{sort}:\mathtt{List}\,\mathbb{N}\rightarrow\mathtt{List}\,\mathbb{N}
\]

and given the first order predicate that 

\[
\mathtt{IsSorted}:\mathtt{List}\,\mathbb{N}\rightarrow*
\]

then it is possible to assert that $\mathtt{sort}$ behaves as expected
with

\[
\lambda x.?:\left(x:\mathtt{List}\,\mathbb{N}\right)\rightarrow\mathtt{IsSorted}\left(\mathtt{sort}x\right)
\]

this term will act like any other term at runtime, given a list input
it will verify that the $\mathtt{sort}$ function correctly handles
that input, give an error, or non-terminate.

Additionally this would allow simple prototyping form first order
specification. For instance,

\begin{align*}
data\,\mathtt{Mult} & :\mathbb{N}\rightarrow\mathbb{N}\rightarrow\mathbb{N}\rightarrow*\,where\\
\mathtt{base} & :\left(x:\mathbb{N}\right)\rightarrow\mathtt{Mult}0\,x\,0\\
\mathtt{suc} & :\left(x\,y\,z:\mathbb{N}\right)\rightarrow\mathtt{Mult}\,x\,y\,z\rightarrow\mathtt{Mult}\,\left(1+x\right)\,y\,(y+z)
\end{align*}

can be used to prototype

\[
\mathtt{div}=\lambda x.\lambda y.\mathtt{fst}\left(?:\sum z:\mathbb{N}.\mathtt{Mult}x\,y\,z\right)
\]

The term search is made subtly easier by the the dynamic equality,
otherwise examples like 

\[
?:\sum f:\mathbb{N}\rightarrow\mathbb{N}.\mathtt{Id}\left(f,\lambda x.x+1\right)\&\mathtt{Id}\left(f,\lambda x.1+x\right)
\]

which would require resolving definitional behavior. Using dynamic
equality it is possible only consider the extensional behavior of
functions.

Though the proof search is currently primitive, better search methods
could be incorporated in future work.

\subsection{Prior work}

Proof search is often used for static term generation in dependently
typed languages (for instance Coq tactics). A first order theorem
prover is attached to Agda in \cite{norell2007towards}.

Twelf made use of runtime proof search but the underling theory cannot
be considered full spectrum.

\section{Testing dependent programs}

Both dynamic equalities and dynamic proof search vastly weaken the
guarantees of normal dependent type systems. Programmers still would
like a evidence of correctness, even while they intend to provide
full proofs of properties in the future. However, there are few options
available in full spectrum dependently typed languages aside from
costly and sometimes unconstructable proofs.

The mainstream software industry has similar needs for evidence of
correctness, and has made use of testing done in a separate execution
phase. Given the rich and precise specifications that dependent types
provide it is possible to improve on the hand crafted tests used by
most of the industry. Instead we can use a type directed symbolic
execution, to run questionable equalities over concrete values and
engage and precompute the searched proof terms. Precomputed proof
terms can be cached, so that exploration is not too inefficient in
the common case of repeating tests at regular intervals of code that
is mostly the same. Precomputed terms can be made available at runtime,
covering for the inefficient search procedure. 

Finally future work can add more advanced methods of testing and proof
generation. This architecture should make it easier to add more advanced
exploration and search without changing the underlining definitional
behavior.

\subsection{Prior work}

\subsubsection{Symbolic Execution}

Most research for Symbolic Execution targets popular languages (like
C) and uses SMT solvers to efficiently explore branches that depend
on base types. Most work does not support higher order functions or
makes simplifying assumptions about the type system. There are however
some relevant papers:
\begin{itemize}
\item \cite{10.1145/3314221.3314618} presents a symbolic execution engine
supporting Haskell's lazy execution and type system. Higher order
functions are not handled
\item The draft work\cite{2006.11639}, handles higher order functions as
and inputs provides a proof of completeness
\item Symbolic execution for higher order functions for a limited untyped
variant of PCF is described in \cite{nguyen2017higher}
\end{itemize}

\subsubsection{Testing dependent types}

There has been a long recognized need for testing in addition to proving
in dependent type systems
\begin{itemize}
\item In \cite{dybjer2003combining} a QuickCheck style framework was added
to an earlier version of Agda
\item QuickChick\footnote{https://github.com/QuickChick/QuickChick} \cite{denes2014quickchick}\cite{lampropoulos2017generating,lampropoulos2017beginner,lampropoulos2018random}
is a research project to add testing to Coq. However testing requires
building types classes that establish the properties needed by the
testing framework such as decidable equality. This is presumably out
of reach of novice Coq users.
\end{itemize}
\bibliographystyle{plain}
\bibliography{C:/icfp/dtest/extended-abstract/dtest}

\end{document}
